\documentclass{rapportENSET}

% Encoding + language
\usepackage[utf8]{inputenc}
\usepackage[T1]{fontenc}
\usepackage[french]{babel}

% Core packages
\usepackage{graphicx}
\usepackage{float}
\usepackage{xcolor}
\usepackage{geometry}
\usepackage{amsmath, amssymb}
\usepackage{enumitem}
\usepackage{fancyhdr}
\usepackage{tikz}
\usepackage{tcolorbox}
\usepackage{caption}
\usepackage{pifont}
\usepackage{listings}
\usepackage{minted}

% -----------------------------------------------------------
% Colors
% -----------------------------------------------------------

\definecolor{winblue}{RGB}{0,120,215}
\definecolor{lightgray}{RGB}{240,240,240}

\definecolor{javared}{rgb}{0.6,0,0}
\definecolor{javagreen}{rgb}{0.25,0.5,0.35}
\definecolor{javapurple}{rgb}{0.5,0,0.35}
\definecolor{javadocblue}{rgb}{0.25,0.35,0.75}
\definecolor{codebackground}{rgb}{0.95,0.95,0.92}

\definecolor{bg}{rgb}{0.95,0.95,0.95} % background for custom boxes

% -----------------------------------------------------------
% Custom item bullets
% -----------------------------------------------------------

\newcommand{\bulletpoint}{\textcolor{winblue}{\textbullet}}
\newcommand{\crossmark}{\textcolor{javared}{\ding{55}}}
\newcommand{\arrowpoint}{\textcolor{winblue}{\ding{226}}}

\renewcommand{\labelitemi}{\bulletpoint}
\renewcommand{\labelitemii}{\arrowpoint}
\renewcommand{\labelitemiii}{\checkmark}

% -----------------------------------------------------------
% Listings configuration (final merged version)
% -----------------------------------------------------------

\lstset{
    language=Java,
    basicstyle=\ttfamily\small,
    keywordstyle=\color{javapurple}\bfseries,
    stringstyle=\color{javared},
    commentstyle=\color{javagreen},
    morecomment=[s][\color{javadocblue}]{/**}{*/},
    numbers=left,
    numberstyle=\tiny\color{black},
    stepnumber=1,
    numbersep=10pt,
    tabsize=4,
    showspaces=false,
    showstringspaces=false,
    backgroundcolor=\color{codebackground},
    frame=single,
    captionpos=b,
    breaklines=true,
    breakatwhitespace=false,
    escapeinside={(*@}{@*)}
}

% -----------------------------------------------------------
% Manual French accents (kept for compatibility)
% -----------------------------------------------------------

\newcommand{\eacute}{\'{e}}
\newcommand{\egrave}{\`{e}}
\newcommand{\ecirc}{\^{e}}
\newcommand{\ccedil}{\c{c}}
\newcommand{\agrave}{\`{a}}
\newcommand{\acirc}{\^{a}}
\newcommand{\icirc}{\^{i}}
\newcommand{\ocirc}{\^{o}}
\newcommand{\ucirc}{\^{u}}
\newcommand{\Eacute}{\'{E}}
\newcommand{\Ccedil}{\c{C}}

% -----------------------------------------------------------

\title{Rapport Enset - Template}

\begin{document}

%----------- Informations du rapport ---------

\titre{TD1 : Diagramme de cas d'utilisation} 
\sujet{Module : Conception Orienté Objet}

\enseignant{Mme. Sara \textsc{RETAL}}
\eleves{Yasser \textsc{Namez}}

%----------- Initialisation -------------------

\fairemarges
\fairepagedegarde
\tabledematieres

%------------ Corps du rapport ----------------


\section{Exercice 1}

\subsection*{Description des éléments d'un diagramme de cas d'utilisation}

Un diagramme de cas d'utilisation UML est composé des éléments suivants :

\begin{itemize}[label=\textbullet]
    \item \textbf{Acteurs} : Représentés par des pictogrammes (bonhommes), ils désignent les utilisateurs ou systèmes externes qui interagissent avec le système.
    \item \textbf{Cas d'utilisation} : Représentés par des ellipses, ils décrivent les fonctionnalités du système.
    \item \textbf{Frontière du système} : Rectangle qui délimite le périmètre du système étudié.
    \item \textbf{Relations} :
    \begin{itemize}[label=\textasciimacron]
        \item \textit{Association} : Ligne simple entre acteur et cas d'utilisation
        \item \textit{Include} : Relation de dépendance obligatoire (flèche pointillée)
        \item \textit{Extend} : Relation de dépendance optionnelle (flèche pointillée)
        \item \textit{Généralisation} : Héritage entre acteurs ou cas d'utilisation (flèche pleine)
    \end{itemize}
\end{itemize}

\section{Exercice 2 : Système de gestion des notes}
\begin{itemize}[label=\textbullet]
    \item \textbf{Acteur pour la saisie des notes :} \emph{Enseignant}.  
    Cas d'utilisation : \emph{Saisir notes}.
    
    \item \textbf{Consultation des notes par les enseignants :}  
    Cas d'utilisation supplémentaires :  
    \begin{itemize}[label=\textasciimacron]
        \item \emph{Consulter notes par examen}
        \item \emph{Consulter notes par élève}
    \end{itemize}

    \item \textbf{Directeur :} ce n'est pas un nouvel acteur.  
    Le directeur est un \emph{spécialisation de l'acteur Enseignant}.

    \item \textbf{Parent :} c'est un \emph{nouvel acteur}.  
    Cas d'utilisation : \emph{Consulter notes de son enfant}.

    \item \textbf{Édition et impression des notes par le directeur :}  
    Cas d'utilisation à ajouter :  
    \begin{itemize}[label=\textbullet]
        \item \emph{Éditer notes par matière}
        \item \emph{Éditer notes par classe ou niveau}
        \item \emph{Éditer bulletin semestriel}
        \item \emph{Imprimer bulletin}
    \end{itemize}
\end{itemize}

\begin{figure}[H]
\centering
\includegraphics[width=0.65\textwidth]{UseCaseDiagram1.png}
\caption{Diagramme de cas d'utilisation n°1}
\label{fig:usecase1}
\end{figure}

\section{Exercice 3 : Système de bibliothèque}

\begin{enumerate}
    \item Un abonné ne peut emprunter que 3 livres, 3 CD et 2 revues simultanément.
    \item Un abonné ne peut réserver si des documents ne sont pas rendus dans les délais (2 semaines).
    \item Réservation annulée si documents non récupérés sous 24h.
    \item La bibliothécaire vérifie le numéro d'abonné et enregistre le prêt.
\end{enumerate}

\begin{figure}[H]
\centering
\includegraphics[width=0.65\textwidth]{UseCaseDiagram2.png}
\caption{Diagramme de cas d'utilisation n°2}
\label{fig:usecase2}
\end{figure}

\section{Étude de Cas 1 : GAB (Guichet Automatique Bancaire)}

\subsection{Définition du système}
Système de Guichet Automatique Bancaire permettant aux porteurs de cartes bancaires d'effectuer des opérations bancaires automatisées 24h/24.

\subsection{Liste des acteurs}
\begin{itemize}[label=\textbullet]
    \item \textbf{Porteur de carte visa} : Client avec carte bancaire internationale
    \item \textbf{Porteur de carte de la banque} : Client avec carte de crédit de la banque
    \item \textbf{Technicien de maintenance} : Personnel technique
    \item \textbf{Système d'autorisation inter-banque} : Acteur secondaire
    \item \textbf{Système d'Information de la banque} : Acteur secondaire
\end{itemize}

\subsection{3. Diagramme de cas d'utilisation}

\begin{figure}[H]
\centering
\includegraphics[width=0.8\textwidth]{UseCaseDiagram3.png}
\caption{Diagramme de cas d'utilisation n°3}
\label{fig:usecase3}
\end{figure}

\section{Étude de Cas 2 : Caisse de supermarché}

\subsection{Acteurs}
\begin{itemize}[label=\textbullet]
    \item \textbf{Caissier} : Opérateur de la caisse
    \item \textbf{Client} : Personne effectuant l'achat
    \item \textbf{Système d'autorisation bancaire} : Acteur secondaire pour paiement carte
\end{itemize}

\subsection{Diagramme de cas d'utilisation}

\begin{figure}[H]
\centering
\includegraphics[width=0.7\textwidth]{UseCaseDiagram4.png}
\caption{Diagramme de cas d'utilisation n°4}
\label{fig:usecase4}
\end{figure}

\section{Étude de Cas 3 : Gestion de bibliothèque}

\subsection{Acteurs}
\begin{itemize}[label=\textbullet]
    \item \textbf{Acheteur} : Service des achats d'exemplaires
    \item \textbf{Responsable inscription} : Gère l'inscription des emprunteurs
    \item \textbf{Responsable prêts} : Enregistre les emprunts
    \item \textbf{Responsable retours} : Gère les retours et relances
    \item \textbf{Emprunteur} : Utilisateur de la bibliothèque
    \item \textbf{Éditeur} : Acteur secondaire (fournisseur)
\end{itemize}

\subsection{Diagramme de cas d'utilisation}
\begin{figure}[H]
\centering
\includegraphics[width=0.8\textwidth]{UseCaseDiagram5.png}
\caption{Diagramme de cas d'utilisation n°5}
\label{fig:usecase5}
\end{figure}

\end{document}
