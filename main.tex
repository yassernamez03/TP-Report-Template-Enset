\documentclass{rapportENSET}

% Encoding + language
\usepackage[utf8]{inputenc}
\usepackage[T1]{fontenc}
\usepackage[french]{babel}

% Core packages
\usepackage{graphicx}
\usepackage{float}
\usepackage{xcolor}
\usepackage{geometry}
\usepackage{amsmath, amssymb}
\usepackage{enumitem}
\usepackage{fancyhdr}
\usepackage{tikz}
\usepackage{tcolorbox}
\usepackage{caption}
\usepackage{pifont}
\usepackage{listings}
\usepackage{minted}

% -----------------------------------------------------------
% Colors
% -----------------------------------------------------------

\definecolor{winblue}{RGB}{0,120,215}
\definecolor{lightgray}{RGB}{240,240,240}

\definecolor{javared}{rgb}{0.6,0,0}
\definecolor{javagreen}{rgb}{0.25,0.5,0.35}
\definecolor{javapurple}{rgb}{0.5,0,0.35}
\definecolor{javadocblue}{rgb}{0.25,0.35,0.75}
\definecolor{codebackground}{rgb}{0.95,0.95,0.92}

\definecolor{bg}{rgb}{0.95,0.95,0.95} % background for custom boxes

% -----------------------------------------------------------
% Custom item bullets
% -----------------------------------------------------------

\newcommand{\bulletpoint}{\textcolor{winblue}{\textbullet}}
\newcommand{\crossmark}{\textcolor{javared}{\ding{55}}}
\newcommand{\arrowpoint}{\textcolor{winblue}{\ding{226}}}

\renewcommand{\labelitemi}{\bulletpoint}
\renewcommand{\labelitemii}{\arrowpoint}
\renewcommand{\labelitemiii}{\checkmark}

% -----------------------------------------------------------
% Listings configuration (final merged version)
% -----------------------------------------------------------

\lstset{
    language=Java,
    basicstyle=\ttfamily\small,
    keywordstyle=\color{javapurple}\bfseries,
    stringstyle=\color{javared},
    commentstyle=\color{javagreen},
    morecomment=[s][\color{javadocblue}]{/**}{*/},
    numbers=left,
    numberstyle=\tiny\color{black},
    stepnumber=1,
    numbersep=10pt,
    tabsize=4,
    showspaces=false,
    showstringspaces=false,
    backgroundcolor=\color{codebackground},
    frame=single,
    captionpos=b,
    breaklines=true,
    breakatwhitespace=false,
    escapeinside={(*@}{@*)}
}

% -----------------------------------------------------------
% Manual French accents (kept for compatibility)
% -----------------------------------------------------------

\newcommand{\eacute}{\'{e}}
\newcommand{\egrave}{\`{e}}
\newcommand{\ecirc}{\^{e}}
\newcommand{\ccedil}{\c{c}}
\newcommand{\agrave}{\`{a}}
\newcommand{\acirc}{\^{a}}
\newcommand{\icirc}{\^{i}}
\newcommand{\ocirc}{\^{o}}
\newcommand{\ucirc}{\^{u}}
\newcommand{\Eacute}{\'{E}}
\newcommand{\Ccedil}{\c{C}}

% -----------------------------------------------------------

\title{Rapport Enset - Template}

\begin{document}

%----------- Informations du rapport ---------

\titre{TP4 - Mise en place de Docker et Cockpit sur une instance Ubuntu EC2} 
\sujet{Module : Virtualisation et Cloud computing}

\enseignant{Pr. Azeddine \textsc{KHIAT}}
\eleves{Yasser \textsc{Namez}}

%----------- Initialisation -------------------

\fairemarges
\fairepagedegarde
\tabledematieres

%------------ Corps du rapport ----------------

\section{Introduction}
Ce rapport présente la réalisation du TP4 ayant pour objectif la mise en place d'un environnement de conteneurisation Docker et d'une interface d'administration Cockpit sur une instance Ubuntu hébergée sur AWS EC2.

\subsection{Objectifs du TP}
\begin{itemize}[label=\textbullet]
    \item Créer et configurer une machine virtuelle Ubuntu sur AWS EC2
    \item Installer et utiliser Docker pour la conteneurisation d'applications
    \item Déployer Cockpit, une interface web d'administration Linux
    \item Superviser et gérer les conteneurs et services via une interface graphique
\end{itemize}

\subsection{Technologies utilisées}
\begin{itemize}[label=\textbullet]
    \item \textbf{AWS EC2} : Service de machines virtuelles dans le cloud
    \item \textbf{Ubuntu 22.04 LTS} : Système d'exploitation Linux
    \item \textbf{Docker} : Plateforme de conteneurisation
    \item \textbf{Cockpit} : Interface web d'administration système
\end{itemize}


\section{Réalisation pratique}

\subsection{Étape 1 : Création de l'instance EC2}

\subsubsection{Configuration de l'instance}

J'ai créé une instance EC2 avec les paramètres suivants :
\begin{itemize}[label=\textbullet]
    \item \textbf{AMI :} Ubuntu Server 22.04 LTS
    \item \textbf{Type d'instance :} t2.micro
    \item \textbf{Clé SSH :} tp4-docker-key.pem
    \item \textbf{Groupe de sécurité :} Ports 22 (SSH) et 9090 (Cockpit) ouverts
\end{itemize}

\begin{figure}[H]
    \centering
    \includegraphics[width=0.8\textwidth]{screenshot1_ec2_creation.png}
    \caption{Création de l'instance EC2 sur AWS}
    \label{fig:ec2_creation}
\end{figure}

\subsubsection{Configuration du groupe de sécurité}
Les règles de sécurité configurées permettent l'accès SSH et Cockpit :

\begin{figure}[H]
    \centering
    \includegraphics[width=0.8\textwidth]{screenshot2_security_group.png}
    \caption{Configuration du groupe de sécurité}
    \label{fig:security_group}
\end{figure}

\newpage

\subsection{Étape 2 : Connexion à l'instance}

Commande utilisée pour se connecter à l'instance :

\begin{lstlisting}[language=bash]
ssh -i "KEY_TP4_VM_LINUX.pem" ubuntu@ec2-3-84-104-201.compute-1.amazonaws.com
\end{lstlisting}

\begin{figure}[H]
    \centering
    \includegraphics[width=0.8\textwidth]{screenshot3_ssh_connection.png}
    \caption{Connexion SSH réussie à l'instance Ubuntu}
    \label{fig:ssh_connection}
\end{figure}

\subsection{Étape 3 : Mise à jour du système}

Commandes exécutées :

\begin{lstlisting}[language=bash]
sudo apt update && sudo apt upgrade -y
\end{lstlisting}

\begin{figure}[H]
    \centering
    \includegraphics[width=0.8\textwidth]{screenshot4_system_update.png}
    \caption{Mise à jour du système Ubuntu}
    \label{fig:system_update}
\end{figure}

\subsection{Étape 4 : Installation de Docker}

\subsubsection{Installation}
Commandes exécutées :

\begin{lstlisting}[language=bash]
sudo apt install -y docker.io
sudo systemctl enable docker
sudo systemctl start docker
sudo usermod -aG docker ubuntu
\end{lstlisting}

\subsubsection{Test de Docker}
Après reconnexion, test avec hello-world :

\begin{lstlisting}[language=bash]
docker run hello-world
\end{lstlisting}

\begin{figure}[H]
    \centering
    \includegraphics[width=0.8\textwidth]{screenshot6_docker_hello_world.png}
    \caption{Test de Docker avec hello-world}
    \label{fig:docker_test}
\end{figure}

\subsubsection{Déploiement d'un conteneur Nginx}
Pour démonstration, j'ai déployé un conteneur Nginx :

\begin{lstlisting}[language=bash]
docker run -d --name test-nginx -p 8080:80 nginx
docker ps
\end{lstlisting}

\begin{figure}[H]
    \centering
    \includegraphics[width=0.9\textwidth]{screenshot7_nginx_container.png}
    \caption{Conteneur Nginx en cours d'exécution}
    \label{fig:nginx_container}
\end{figure}


\subsection{Étape 5 : Installation de Cockpit}

Commandes exécutées :

\begin{lstlisting}[language=bash]
sudo apt install -y cockpit
sudo apt install -y cockpit-docker
sudo systemctl enable --now cockpit.socket
\end{lstlisting}

\begin{figure}[H]
    \centering
    \includegraphics[width=0.8\textwidth]{screenshot8_cockpit_install.png}
    \caption{Installation de Cockpit et du plugin Docker}
    \label{fig:cockpit_install}
\end{figure}

\subsection{Étape 6 : Configuration de l'accès Cockpit}

\subsubsection{Définition du mot de passe}
\begin{lstlisting}[language=bash]
sudo passwd ubuntu
\end{lstlisting}

\subsubsection{Vérification du service Cockpit}
\begin{lstlisting}[language=bash]
sudo systemctl status cockpit.socket
\end{lstlisting}

\subsection{Étape 7 : Accès à l'interface Cockpit}

\subsubsection{Page de connexion}
Accès via : \texttt{https://3.84.104.201:9090}

\begin{figure}[H]
    \centering
    \includegraphics[width=0.8\textwidth]{screenshot10_cockpit.png}
    \caption{Page de connexion Cockpit}
    \label{fig:cockpit_login}
\end{figure}

\end{document}
